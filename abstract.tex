% !TEX root = demo.tex
\begin{abstract}
%The prevalence of dirty data presents a fundamental obstacle to modern data-driven applications.
Analysts report spending upwards of 80\% of their time on problems in data cleaning.
The data cleaning process is inherently iterative, with evolving cleaning workflows that 
start with basic exploratory data analysis on small samples of dirty data, then refine analysis with 
more sophisticated/expensive cleaning operators (i.e., crowdsourcing), and finally apply the insights to a full dataset.
While an analyst often knows at a logical level what operations need to be done, they often have 
to manage a large search space of physical operators and parameters.
We present \sys, a system designed to support the iterative development and optimization of data cleaning workflows, especially ones that utilize the crowd.
\sys separates logical operations from physical implementations, and driven by analyst feedback, suggests optimizations and/or replacements to the analysts choice of physical implementation.
To implement this, we highlight research challenges in sampling, in-flight operator replacement, and crowdsourcing. 
We overview the system architecture and these techniques, 
then propose a demonstration designed to showcase how \sys can improve iterative data analysis
and cleaning. 
The code is available at: \url{http://www.sampleclean.org}.
\end{abstract}

