\begin{abstract}
Dealing with dirty data is a fundamental barrier in modern data-driven applications -- 
blindly using results that are derived from dirty data, such as machine learning models 
or analytical queries, can contain hidden, yet significant, errors.
To combat dirty data, analysts can easily spend 80\% or more of their analysis 
time~\cite{kandelvast} attempting to identify and understand the data errors, 
while concurrently building custom scripts or large data cleaning pipelines 
in order to manage and fix the errors.   The domain-specificity of data cleaning
necessitates an interactive feedback loop between the user's changes to the data cleaning 
pipeline, and the data cleaning results.

Existing data cleaning systems are either batch-oriented processing
systems that lack interactivity, or interactive systems that are
designed for a specific data cleaning task (e.g., deduplicating
\ewu{a specific type of data?}, or finding outliers).  In contrast,
we have distilled the core components of existing data cleaning
systems into a small set of logical operators that can be composed
into a data cleaning plan, and designed a that minimizes the feedback
loop and supports dynamic reconfiguration while the plan executes.
The code is available at: \url{http://www.sampleclean.org}.
\end{abstract}

