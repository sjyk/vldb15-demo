\section{Example Applications} 

In this section, we describe introduce some example data cleaning applications built using the \sys language.
%We have an additional operator \textsf{Async} which designates the 
%execution of the operator to be synchronous or asynchronous.


\subsubsection{Entity Resolution}

Corelone query.


\subsubsection{String Extraction}
One of the most common data cleaning operations is delimited extraction.
In our DSL, this can be expressed in the following way:
\begin{lstlisting}
Extract on Data
with Split, `,',
cols=[col1, col2]
\end{lstlisting}

We can also use a format string:
\begin{lstlisting}
Extract on Data
with FmtString, `%s:%s',
cols=[col1, col2]
\end{lstlisting}

\subsubsection{Crowd Entity Resolution}
Crowd-based techniques for entity resolution are increasingly popular e.g Corleone \cite{DBLP:conf/sigmod/GokhaleDDNRSZ14}.
We present an example of expressing a crowd-based technique for entity resolution with the DSL.
Let us suppose we have a database of addresses that we want to deduplicate.
The first step is to group similar rows together and a good similarity metric to use is JaccardSimilarity.
\begin{lstlisting}
SimilarityJoin on Data
with Jaccard, thresh=0.8
\end{lstlisting}
The next step is to filter these pairs using the crowd. 
However, since most pairs will not be duplicates we want to use Active Learning.
Furthermore, since the crowd might be slow, we can add asynchrony.
\begin{lstlisting}
Filter on
( 
 SimilarityJoin on Data
 with Jaccard, thresh=0.8
)
with Crowd, Active, Async
\end{lstlisting}
To resolve the changes, we apply transitive closure at the end that takes the 
longest address:
\begin{lstlisting}
TransitiveClosure (
 Filter on
 ( 
  SimilarityJoin on Data
  with Jaccard, thresh=0.8
 )
 with Crowd, Active
) with Longest
\end{lstlisting}


\subsubsection{Approximate Query Processing}
Oftentimes, the goal of data cleaning is to compute accurate 
statistics of the data set.  The problem with dirty data is unknown bias.
For these queries, \sys supports an approximate query processing interface 
over the data cleaning results view.  The mechanisms are based of prior
SampleClean work~\cite{sampleclean} that allows for unbiased aggregate result estimation during prototyping and debugging.
%It further allows for reduced cleaning costs if the user is interested in only aggregate query processing as that may
%not require cleaning the full data.  

\ewu{show a query, or how to make AQP work}


