\section{Challenges}
Developing \projx system poses significant challenges in both research and engineering. In this section, we will highlight these challenges, and present our ideas to address them.

\subsection{Research Challenges}



\subsubsection{Instant Feedback Loops}

Since data cleaning processing is often time consuming, to avoid users to feel ``left in the dark" during the process, our system can provide instant feedback loops and make users to get a sense that how data cleaning will change the data and affect the analysis results. We achieve this goal by the following two ideas. 

\vspace{.5em}

{\noindent \bf Sampling-and-Cleaning:} \projx allows users to clean a small sample of data, which is typically much cheaper and faster than cleaning the entire data. With the cleaned sample, the system is able to inform users the changes of their original analysis results, with the help of new query processing methods on dirty data~\cite{wang1999sample}. To be specific, suppose one collects a restaurant dataset, and runs some aggregation queries on the dataset, e.g., computing the number of the restaurants in San Fransisco and getting a query result of 12000. But, when looking at the dataset, she finds that there are many duplicate restaurants in the dataset, and some restaurants' locations are misspelled (e.g., S\underline{e}n Fransisco). She can use our system to create a sample of the data, and then apply a data cleaning procedure to the sample. After the sample is cleaned, our system can estimate the impact of data cleaning on her original query results, and return her corrected answers with confidence intervals, e.g., $5000\pm300$. Intuitively, this result means that if the same data-cleaning procedure is applied to the full dataset, the number of the restaurants in San Fransisco will be within $[4700, 5300]$. 

\vspace{.5em}
{\noindent \bf Asynchronous Data Cleaning:} Even cleaning a sample of data, data cleaning can still take a lot of time. This is especially true when it requires humans to clean the data. 






\subsubsection{Hot Swapping Support}

{\noindent \bf Physical Operators:}

\vspace{.5em}

{\noindent \bf Parameters:}





\subsection{Engineering Challenges}
\subsubsection{API Design}

\subsubsection{Crowdsourcing Platform}

\iffalse
\subsection{Research Challenges}

Architecturally, we separate crowd sourcing and automated data cleaning.
There is a crowd server that acts as a layer of indirection between the Spark codebase and crowd sourcing APIs.

\team{Describe Research Challenges.  To what extend can we actually do
optimization given the physical operators?  Unlike relalgebra, the physical operators here are
not interchangable!}



\subsection{Learning Parameters From Example}
In simple cases, it might be easy to use domain knowledge to select and tune physical operators. 
In more complex cases, it might be easier to specify a sample of dirty and clean data instead of the function.
It may also not be feasible to have the crowd clean the entire dataset.
In these cases, we want to learn a statistical model from which we can extrapolate those responses to the rest 
of the data.

In our current implementation of \projx, we pose this learning problem as classification problems.
In general, these parameter functions can be quite complex and this is a simplification of the learning problem.
The choice of classifier and featurization is upto the user. 
We currently support Support Vector Machines and Decision Trees with a featurization library that includes common text processing features.

\vspace{0.5em}

\noindent \textsf{Filter(R, $\mathcal{T}^+$, $\mathcal{T}^-$)}: Given a set of positive training examples $\mathcal{T}^+$ (i.e, $r$ that satisfy the condition) and
negative training examples $\mathcal{T}^-$ (i.e, r that do not satisfy the condition), we learn a classifier that predicts whether a record satisfies the condition. 

\vspace{0.5em}

\noindent \textsf{Extract(R, a, $\mathcal{T}$)} We restrict the learned problem setting to delimited extraction. Given a set of training examples $R(a)$ and the output $v_1,v_2,...,v_k$, we learn a classifier to predict which characters in $R(a)$ are delimiters based on the tokens in the string.

\vspace{1em}

To acquire the samples of clean data, uniform sampling may not be the best strategy.
For example, if there examples of dirty data are very rare, we will not be able to learn a model.
We implement a technique called Active Learning to sample.
Active Learning selects the most informative examples based on the current model so far.
We use an Active Learning algorithm called uncertainty sampling to do this.




\subsection{Inspection}

Data cleaning requires inspection -- user wants to see how the dataset
has been "cleaned"  through the pipeline.  What does that even mean?

\subsection{Dynamic Re-optimization}

Crowd means want to swap in or out operators at run time.


\subsection{Lineage}

\noindent\textbf{Lineage: }
We track the lineage of rows using a primary key.
Users are not allowed to modify this primary key with any operations.
This allows us to apply operations like transitive closure even after projection since we have a unique identifier for each row.

\vspace{0.5em}
\noindent \textbf{Example: } Suppose, we are interested in deduplication of unstructured data. Then, we could apply the following logical operations.
We first apply an \textsf{Extract} operation to extract the unstructured data into columns. If some of the columns are inconsistent in their representation,
we apply \textsf{Project} to those columns that are inconsistent. We can then take a \textsf{SimilarityJoin} to group rows that are similar, and finally
we resolve those differences with \textsf{TransitiveClosure}.
\fi


