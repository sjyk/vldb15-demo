\section{\projx DSL}
We design a DSL for the composition of data cleaning operations.
The general syntax of this language is:
\begin{lstlisting}
<logical operator> on <relations>
	with <physical operators> , <params>
\end{lstlisting}
In this section, we will highlight some key examples.
We have an additional operator \textsf{Async} which designates the 
execution of the operator to be synchronous or asynchronous.

\subsection{String Extraction}
One of the most common data cleaning operations is delimited extraction.
In our DSL, this can be expressed in the following way:
\begin{lstlisting}
Extract on Data
with Split, `,',
cols=[col1, col2]
\end{lstlisting}

We can also use a format string:
\begin{lstlisting}
Extract on Data
with FmtString, `%s:%s',
cols=[col1, col2]
\end{lstlisting}

\subsection{Crowd Entity Resolution}
Crowd-based techniques for entity resolution are increasingly popular e.g Corleone \cite{DBLP:conf/sigmod/GokhaleDDNRSZ14}.
We present an example of expressing a crowd-based technique for entity resolution with the DSL.
Let us suppose we have a database of addresses that we want to deduplicate.
The first step is to group similar rows together and a good similarity metric to use is JaccardSimilarity.
\begin{lstlisting}
SimilarityJoin on Data
with Jaccard, thresh=0.8
\end{lstlisting}
The next step is to filter these pairs using the crowd. 
However, since most pairs will not be duplicates we want to use Active Learning.
Furthermore, since the crowd might be slow, we can add asynchrony.
\begin{lstlisting}
Filter on
( 
 SimilarityJoin on Data
 with Jaccard, thresh=0.8
)
with Crowd, Active, Async
\end{lstlisting}
To resolve the changes, we apply transitive closure at the end that takes the 
longest address:
\begin{lstlisting}
TransitiveClosure (
 Filter on
 ( 
  SimilarityJoin on Data
  with Jaccard, thresh=0.8
 )
 with Crowd, Active
) with Longest
\end{lstlisting}



