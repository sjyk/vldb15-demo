\subsection{Engineering Challenges}
In addition to the numerous research challenges, there have been substantial challenges in engeering \sys.
We highlight two key challenges which are API design and Crowdsourcing.

\subsubsection{API Design}
Many of the research challenges in \sys rely on clear specifications of the input and output behaviors of
the data cleaning operators.
For example, Hot Swapping, is only possible if two physical operators have exactly the same input and output specification.
The engineering challege to to retain modularity and extensibility while restricting operators.
We build a class heirarchy of data cleaning operators which through inheritance and parameterization specify clear input and output behaviors.

One example of this is the Similarity Join operator.
Users might chose complex similarity functions making it hard to understand their semantics for optimization.
We abstract the similarity function into a abstract class called a \textsf{SimilarityFeature}, this class is parameterized
by a single threshold, contains an API to specify whether two records are similar based on the threshold, maintains its
own tokenizer, and has an API to specify available optimization properties such as monotonicity and prefix-filtering.
So if a user proposes an arbitrary similarity function, as long as it is implemented in the API, it can benefit from our
parameter tuning recommendations.

One draw back of an extensive API is that the choices may overwhelm a user.
Many of our API methods and classes come with inherited sensible defaults.
For example, if the user fails to completely specify the semantics of the similarity function through the API, 
the system will avoid optimizations and perform a naive cartesian products and filtering.
These behaviors come naturally through the use of inheritance and polymorphism.

\subsubsection{Crowdsourcing Platform}
The second engineering challenge is designing a platform to serve crowd tasks.
First, working with crowds is inherrently challenging.
Unlike for our automated operators, we have to manage greatly varying crowd worker
response times and quality.
We developed a ``crowd server" to add a layer of indirection to support this.
The crowd server handle serving and processing crowd tasks, including quality control through redudancy.
We aggregate redundant responses using an Expectation-Maximization algorithm.
Only after collecting sufficent confidence for a task, we process that tuple in our pipeline.

The other benefit of a crowd server is that the APIs for Amazon Mechanical Turk (AMT), Crowd Flower, and others greatly vary.
One common feature of these APIs is that they serve tasks through a browser-based interface.
Our crowd server generates tasks in the form of dynamic web pages and process responses through AJAX call backs.
As a result, the same server works for a variety of different crowds including AMT, Crowd Flower, and a custom ``Internal Crowd".