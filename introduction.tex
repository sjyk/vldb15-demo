\section{Introduction}\label{sec:intro}

% why feedback loops exist/are important (cite Joe’s shit in the past 3 years, blinkdb, etc.). highlight domain specificity. Large variety of data cleaning tasks

%
% Genereal comment that data cleaning is important:
%
Dealing with dirty data is a fundamental barrier in modern data-driven applications -- 
blindly using results that are derived from dirty data can lead to hidden, yet significant, errors.
To combat dirty data, analysts can easily spend 80\% or more of their analysis 
time~\cite{kandel2012} attempting to identify and understand the data errors, 
while concurrently building and iteratively refining custom scripts or large data cleaning pipelines 
in order to manage and fix the errors.  

Removing data errors from even a single dataset is 
time consuming is due to the breadth of possible errors that are manifest in the data,
the variety of ways to clean the data, and the dependent on the specific application.
For example, sample-and-clean techniques can compensate for bias in aggregation query 
results without needing to clean the entire dataset, extracting and de-duplicating
restaurant listings for a consumer facing website requires cleaning the entire dataset.
In addition, the user must go through this cleaning process for each of her data sources, as well as
for possible errors introduced as part of earlier data curation steps~\cite{}.
These tasks are infeasible without a general, user-efficient, and scalable data cleaning system.



%The prevalance of data cleaning systems in both the research and industrial communities --
%Corleone does blah, XXX addresses blah. Nadeef does blah -- speak to the importance of a
%data cleaning framework as part of the modern big data ecosystems. \ewu{include open access of data in argument?} \jn{We also need to take a look at data cleaning systems in industry. }


% why existing systems suck aka related work
% 1. have slow feedback loops (dataset-dependence, …)
% 2. solve very specific data-cleaning tasks
The problem of data cleaning is not new -- data cleaning systems have been
explored by both the research and industry communities since the beginning of data management.
Extract-transform-load (ETL) systems~\cite{informatica,talend,nadeef,apachefalcon}
required developers to manually wrote data cleaning rules which were executed as 
long batch jobs.
Projects such as Wrangler~\cite{wrangler,trifacta} and
OpenRefine~\cite{openrefine} observed that data cleaning is highly iterative, and
introduced user-centric spreadsheet-style interfaces that enable the user to inspect a sample of the dataset, 
compose data cleaning sequences using a direct manipulation interface, and apply these sequences over the full dataset.   
More recently, a number of crowd-based~\cite{gokhale2014corleone,stonebraker2013data}
systems have been proposed to reduce the burden of performing the manual cleaning tasks from the data cleaning analyst 
by utilizing crowd workers on systems such as Mechanical Turk or ODesk~\cite{argonaut}.

Unfortunately, existing systems are limited in two key ways.
First, the batch oriented systems result in a very slow feedback loop 
that inhibits the user's ability to rapidly prototype different data cleaning solutions (e.g., tweak and 
try different similarity metrics or sampling techniques based on current data cleaning results).
Second, the existing interactive and iterative systems are designed for very specific cleaning tasks 
(e.g., de-duplication~\cite{gokhale2014corleone,park2014crowdfill,eracer,chen2014integrating}) that may not apply to all of the data 
cleaning operations necessary for a dataset.  The user must then combine and integrate multiple
data cleaning systems in order to clean their data.
Third, these systems do not employ task-specific optimizations
such as tuning similarity threshold parameters or replacing fast-but-inaccurate operators with slower but more accurate ones.
These existing limitations suggest the need for a system that is both general enough to
adapt to a wide range of data cleaning applications, scalable to large datasets, 
and that natively supports high-feedback interactions to enable rapid data cleaning iterations.

To this end, we are currently building \sys\footnote{{\scriptsize \sys uses {\bf C}rowds, {\bf C}ode, and machine {\bf L}earners to deep cleanse dirty data.}}, 
a scalable data cleaning
framework designed with rapid feedback in mind.  Our goal is to
engineer a highly interactive form of data cleaning where the user can easily prototype a data cleaning execution plan,
quickly inspect the quality of the cleaned results at any point in the plan, and seamlessly
tweak the operators and parameters on the fly.

Supporting these capabilities requires a mixture of careful engineering 
as well as tackling several research challenges:

\squishlist
\item {\bf Operator and API design}: We have designed a core set of logical data cleaning operators --
filter, similarity join, extract and sample -- that can be used to
implement operations such as Blocking, and entity matching.  \sys also supports
rule-based, machine learning-based, and crowd-sourcing based physical implementations of the operators.  
This combination of logical and physical designs lets users to easily start the cleaning process
with standard rule-based algorithms, transition to human-workers for machine-difficult 
errors, and incrementally build models to automate the cleaning process.  
In addition, we have developed a large library of common implementations of 
each logical operator that can be easily tuned and used out of the box, as well as a crowd-managment 
framework that supports multiple crowd-sourcing platforms such as Mechanical Turk and ODesk.

\item {\bf Near-instant feedback}: We are investigating multiple approaches to reduce the 
feedback loop throughout the system.  First, nteracting with a crowd is commonly the primary latency bottleneck,
and we are studying techniques such as straggler mitigation~\cite{venkataraman2014power}, worker latency modeling, and
bandit algorithms~\cite{}\ewu{DAN FILL THIS} to consistently and cost effectively retrive crowd results in seconds rather than hours. 
Second, minimizing the time for the {\it first} end-to-end cleaning result is arguably more important than 
end-to-end latency for cleaning the entire dataset.    We are studying techniques to pick samples that are
most likely to be cleaned by the plan (and consequently be interesting to the user.)

\item {\bf Hot Swapping}: In contrast to batch-oriented data cleaning, rapid iteration necessitates the
ability to tune an operator parameter or reconfigure the sequence of cleaning operators as the cleaning
plan is running.   We are also studying the conditions for when the partial results for one
cleaning plan can be safely used in a similar but reconfigured plan.  We are using this hot-swapping primitive
to implement a cleaning plan optimizer that continually refines the cleaning plan based on user-defined latency-quality
tradeoffs.

\squishend


In this demonstration, we will run an entity resolution plan on two restaurant datasets, and
show how \sys can be used to 1) specify and execute a data cleaning plan for the first dataset using our domain specific
language, 2) quickly clean a sample of the first dataset to characterize how data is cleaned
in the end-to-end plan, 3) apply the cleaning plan to the second dataset and show that the cleaned dataset
is suboptimal, 4) incrementally refine the plan by changing an operator's parameter (e.g., similarity threshold)
or its physical implementation.
The fourth interaction lets the participants act as a human optimizer and inspect the effects of different physical plans.
Users can then execute the plan over a live crowd that uses the audience as workers, or a simulated crowd
that uses pre-collected crowd responses.    The dashboard (Figure~\ref{screenshot-rec} also provides a live inspection
interface to view the status of the cleaning plan as it executes.



%Our contributions/requirements
%different ways to tighten the feedback loop:
%end-to-end latency/cost (operator optimization)
%looking versus touching
%Adding introspection (more points of observation)
%hot-swapping (more points of changing plans)
%We have built an end-to-end data cleaning framework with these requirements in mind. (... things we do …) (... engineering contributions …).
%In this demonstration, we highlight the benefits of improving feedback loops for data cleaning using X datasets by optimizing a data cleaning pipeline for one data set/cleaning task, then quickly fitting the pipeline to another dataset.


\if{0}

\jn{Honestly, I didn't quite buy declarativity of the system. In my opinion, data cleaning is so domain specific. It's hard to make it declarative. For a given domain, people may need to write their own data cleaning system. There is a lack of a data cleaning framework that they can build based on. This motivates us to develop such framework. 


We analyze a large variety of domain specific data cleaning systems, and identify several key components: declarative data cleaning operators (e.g., similarity joins), active learning, and crowd/expert sourcing platforms that they require. In our framework, we abstract these components, and implement them in a general way. 

We mainly address two challenges: extensibility and scalability. For the former one, we came up with a nice data-cleaning pipeline API, which people can easily use to compose their own data cleaning tasks. For the latter one, we address it in two aspects: Sampling + Asynchrony.}

\ewu{That's fair, will need to address why a framework is necessary and what benefits it provides.  I think a framework is the correct pitch, hard to sell a set of operators.  Are the above challenges -- extensibility and scalability -- actually difficult?  Worried it's straightforward application of existing techniques.}



In contrast, our work is based on the observation that the majority of data cleaning workflows
can be decomposed into a small set of logical operations (in addition to traditional database operators):
filter based on constraints, extract new fields from existing data, and a similarity join to match
similar or duplicate records. \ewu{quickly validate why this observation holds.} \jn{Yes! I also found that Sec 2.3 has more operators than you describe here.}  
By designing a system around these core operators, we can provide a vast library of physical  
data cleaning operators that span the range of algorithmic, machine learning, and human computation-based
implementations that are necessary practical data cleaning pipelines.   \ewu{Describe live inspection as 
a core feature or is it too easy?}

Designing such a system requires tackling several design challenges:

\begin{enumerate}
\item Speed
\item Quality
\item API Design/extensibilty
\end{enumerate}



We have implemented an initial version of \sys on top of the AMPLab Spark stack, which provides us 
with access to its advanced distributed processing and machine learning features.  Our goal for the current
version is to implement the core mechanisms for declarative specification of the
data cleaning pipeline, solidify the API design, and incle support for, and implementations of,
multiple classes of physical data cleaning operators.


\fi



\if{0}

Cleaning, pre-processing, and formatting data is a required first step in any data analytics pipeline.
However, despite this importance, large-scale data analytics platforms such as Spark or Hadoop lack integrated data cleaning frameworks.
There are a few challenges in building a general purpose data cleaning framework: (1) data cleaning is often
domain specific and requires specialized software targeted at one or a handful of data sources \cite{wang1999sample}, (2) data cleaning is often 
expensive as it increasingly involves human effort via crowdsourcing or experts \cite{DBLP:conf/sigmod/GokhaleDDNRSZ14}, and (3) learning how to clean dirty data from examples
is often hard without a greatly restricted set of operators \cite{DBLP:conf/uist/GuoKHH11}.

We address this problem in \projx by designing a Spark library of composable and scalable data cleaning primitives.
\projx abstracts the logical data cleaning operators: Extraction, Similarity Join, Filtering, from the physical implementation i.e, Rule, Crowd, or Machine Learning.
We interface these primitives through a DSL with which a user can build data cleaning operators that suit their needs.
\projx provides transparent optimizations for each of the components and their composition.
In this demonstration proposal, we present \projx and highlight some of its key features.
While there are many existing systems that do one aspect of data cleaning and transformation (e.g Entity Resolution or Extraction), 
many real world data cleaning tasks have multiple types of errors.
Composing disparate systems can lead to complex code and inefficiencies at scale.
With \projx, we hope to design a set of optimized composable primitives that span a large space of data cleaning tasks.

The first key feature of \projx is that it provides optimized distributed implementations 
of the physical data cleaning operators.
For example, a key step in many deduplication algorithms is a Similarity Join which finds all pairs of records that are within some similarity threshold.
A naive implementation of a Similarity Join would apply a similarity function to all pairs of records.
However, in \projx, we provide optimized implementations of certain common similarity functions (e.g Jaccard, Edit Distance, etc.) that allow for 
a combined broadcast join and prefix filtering which intelligently skips pairs of records using a broadcasted inverted index.

Another feature of \projx is managing the latency and the scale problems of crowd-based data cleaning. 
Crowdsourcing is increasingly prevalent in data cleaning, and \projx provides physical crowd-based implementations
of the logical operators.
However, crowds work at a different latency and scale point in comparison to distributed analytics platforms.
To address the latency problem, we build asychrony into the system.
The user can query intermediate results at any time as crowd responses stream in.
To address the scale issue, \projx provides sampling primitives.
The glue that ties all of the crowd components together is a Machine Learning technique called Active Learning.
As we collect more and more crowd responses, we learn a model that predicts these responses to apply it on the uncleaned data.
Active Learning selects the most informative questions to ask the crowd.

Finally, \projx provides an approximate query processing (AQP) framework.
With slow asynchronous data cleaning algorithms as in crowdsourcing, we need 
to define clear semantics for the intermediate results.
Our AQP framework uses the algorithms proposed in \cite{wang1999sample}, to estimate and bound early results.
It is also common for data scientists to prototype expensive data cleaning pipelines on samples and AQP allows quick evaluation of
aggregate query results on a cleaned sample.

\subsection{Demonstration Scenario}
\reminder{TODO}

\fi




\if{0}
Consider for example, the ability to rapidly understand the types of errors that are present, as well as prevalance of 
these errors is cruicial.



Before an organization can use a new dataset as part of their analysis pipeline
(e.g., to build complex learning models or answer analyst queries)
the errors in the dataset need to be removed in order to ensure accurate conclusions.  

Modern data-driven organizations rely on the ability to ingest and generate large data sets from 
disparate sources, and combine the data together to build complex models or answer analytical questions.  
For example, a restaurant review website may collect restaurant listings by scraping data from webpages or purchasing them from external sources, and
restaurant visitation information for sources such as OpenTable or FourSquare, and aggregate the data to
model user eating habits.  
The set of cleaning tasks necessary for each of these data sources is highly domain and application specific,
and oftentimes the developer is concurrently trying to clean the data source as well as understand its properties.


Oftentimes, these data sources have data quality issues that require a complex data cleaning pipeline -- 
e.g., data extraction, re-formatting, identification and fixes of missing or incorrect values,
and removal of redundant information -- before the data is useable by downstream processes.
Data sources are often domain specific and new for the data analyst, 
As datasets continue to grow, and organizations make use of mure and more datasets, the ability to
rapidly clean the data is more important.  
\fi
