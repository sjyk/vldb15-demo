\section{Introduction}\label{sec:intro}
Cleaning, pre-processing, and formatting data is a required first step in any data analytics pipeline.
However, despite this importance, large-scale data analytics platforms such as Spark or Hadoop lack integrated data cleaning frameworks.
There are a few challenges in building a general purpose data cleaning framework: (1) data cleaning is often
domain specific and requires specialized software targeted at one or a handful of data sources \cite{wang1999sample}, (2) data cleaning is often 
expensive as it increasingly involves human effort via crowdsourcing or experts \cite{DBLP:conf/sigmod/GokhaleDDNRSZ14}, and (3) learning how to clean dirty data from examples
is often hard without a greatly restricted set of operators \cite{DBLP:conf/uist/GuoKHH11}.

We address this problem in \projx by designing a Spark library of composable and scalable data cleaning primitives.
\projx abstracts the logical data cleaning operators: Extraction, Similarity Join, Filtering, from the physical implementation i.e, Rule, Crowd, or Machine Learning.
We interface these primitives through a DSL with which a user can build data cleaning operators that suit their needs.
\projx provides transparent optimizations for each of the components and their composition.
In this demonstration proposal, we present \projx and highlight some of its key features.
While there are many existing systems that do one aspect of data cleaning and transformation (e.g Entity Resolution or Extraction), 
many real world data cleaning tasks have multiple types of errors.
Composing disparate systems can lead to complex code and inefficiencies at scale.
With \projx, we hope to design a set of optimized composable primitives that span a large space of data cleaning tasks.

The first key feature of \projx is that it provides optimized distributed implementations 
of the physical data cleaning operators.
For example, a key step in many deduplication algorithms is a Similarity Join which finds all pairs of records that are within some similarity threshold.
A naive implementation of a Similarity Join would apply a similarity function to all pairs of records.
However, in \projx, we provide optimized implementations of certain common similarity functions (e.g Jaccard, Edit Distance, etc.) that allow for 
a combined broadcast join and prefix filtering which intelligently skips pairs of records using a broadcasted inverted index.

Another feature of \projx is managing the latency and the scale problems of crowd-based data cleaning. 
Crowdsourcing is increasingly prevalent in data cleaning, and \projx provides physical crowd-based implementations
of the logical operators.
However, crowds work at a different latency and scale point in comparison to distributed analytics platforms.
To address the latency problem, we build asychrony into the system.
The user can query intermediate results at any time as crowd responses stream in.
To address the scale issue, \projx provides sampling primitives.
The glue that ties all of the crowd components together is a Machine Learning technique called Active Learning.
As we collect more and more crowd responses, we learn a model that predicts these responses to apply it on the uncleaned data.
Active Learning selects the most informative questions to ask the crowd.

Finally, \projx provides an approximate query processing (AQP) framework.
With slow asynchronous data cleaning algorithms as in crowdsourcing, we need 
to define clear semantics for the intermediate results.
Our AQP framework uses the algorithms proposed in \cite{wang1999sample}, to estimate and bound early results.
It is also common for data scientists to prototype expensive data cleaning pipelines on samples and AQP allows quick evaluation of
aggregate query results on a cleaned sample.

\subsection{Demonstration Scenario}
\reminder{TODO}







